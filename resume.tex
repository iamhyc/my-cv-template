%%%%%%%%%%%%%%%%%%%%%%%%%%%%%%%%%%%%%%%%%%%%%%%%%%%%%%%%%%%%%%%%%%%%%%%%%%%%%%%%
% Medium Length Graduate Curriculum Vitae
% LaTeX Template
% Version 1.2 (3/28/15)
%
% This template has been downloaded from:
% http://www.LaTeXTemplates.com
%
% Original author:
% Rensselaer Polytechnic Institute 
% (http://www.rpi.edu/dept/arc/training/latex/resumes/)
%
% Modified by:
% Daniel L Marks <xleafr@gmail.com> 3/28/2015
% 
% Further modified by:
% Rohan Bavishi <rohan.bavishi95@gmail.com> 9/20/2016
%
% Important note:
% This template requires the simple_style.cls file to be in the same directory 
% as the .tex file. The res.cls file provides the resume style used for 
% structuring the document.
%
%%%%%%%%%%%%%%%%%%%%%%%%%%%%%%%%%%%%%%%%%%%%%%%%%%%%%%%%%%%%%%%%%%%%%%%%%%%%%%%%

%-------------------------------------------------------------------------------
%	PACKAGES AND OTHER DOCUMENT CONFIGURATIONS
%-------------------------------------------------------------------------------

%%%%%%%%%%%%%%%%%%%%%%%%%%%%%%%%%%%%%%%%%%%%%%%%%%%%%%%%%%%%%%%%%%%%%%%%%%%%%%%%
% You can have multiple style options the legal options ones are:
%
%   centered:	the name and address are centered at the top of the page 
%				(default)
%
%   line:		the name is the left with a horizontal line then the address to
%				the right
%
%   overlapped:	the section titles overlap the body text (default)
%
%   margin:		the section titles are to the left of the body text
%		
%   11pt:		use 11 point fonts instead of 10 point fonts
%
%   12pt:		use 12 point fonts instead of 10 point fonts
%
%%%%%%%%%%%%%%%%%%%%%%%%%%%%%%%%%%%%%%%%%%%%%%%%%%%%%%%%%%%%%%%%%%%%%%%%%%%%%%%%

\documentclass[mm]{simple_style}  

% Default font is the helvetica postscript font
\usepackage{helvet}
\usepackage{hyperref}
\usepackage{url}
\usepackage{xcolor}
\hypersetup {
    colorlinks=true,
    linkcolor=colorlink,
    filecolor=magenta,      
    urlcolor=colorlink,
}
\usepackage[left=0.7in, right=2in, top=0.9in]{geometry}

% Increase text height
\textheight=700pt

\begin{document}

%-------------------------------------------------------------------------------
%	NAME AND ADDRESS SECTION
%-------------------------------------------------------------------------------
\name{Yuncong Hong}
\qualification{PhD Candidate, Computer Science, The University of Hong Kong}
\emailone{ychong@cs.hku.hk}
\emailtwo{hongyc@mail.sustech.edu.cn}
\website{https://resume.sudofree.xyz}{ \url{resume.sudofree.xyz} }
\github{https://github.com/iamhyc}{ \url{github.com/iamhyc} }
\phone{\texttt{+}86 15811823462}
\address{
    1088 Xueyuan Avenue, \\
    Shenzhen 518055, P.R. China
}
%-------------------------------------------------------------------------------

\begin{resume}

%-------------------------------------------------------------------------------
%	EDUCATION SECTION
%-------------------------------------------------------------------------------
\begin{section}{Education}
    \cusemph{Southern University of Science and Technology (SUSTech)} \\ %, China
    {\sl Bachelor of Engineering} in \cusemph{Communication Engineering}, \timeline{Sep '14 - Jul '18}\\
    \cusemph{GPA: 3.77/4.00} (Major)~\cusemph{Ranking: 3/30}
    
    \cusemph{The University of Hong Kong (HKU)} \\ %, Hong Kong, China
    {\sl Doctor of Philosophy} in \cusemph{Computer Science}, \timeline{Sep '18 - Jul '22} (expected)\\
\end{section}
\sectionline

%-------------------------------------------------------------------------------
%      RESEARCH INTERESTS && SKILLS 
%-------------------------------------------------------------------------------
\begin{section}{Research Interests}
  Edge Computing, Application-Centric Network Optimization, and Applications of Markov Decision Process (MDP).
\end{section}

\begin{section}{Research Skills}
    \cusemph{Core Courses}: Convex Optimization, Polynomial Optimization, Stochastic Process, Wireless Communications, and Information Theory.
    
    \cusemph{Programming Languages}: Python, Rust, C/C\texttt{++}, Nodejs, TypeScript, and LabVIEW.
    % \LaTeX, Java, HTML/(S)CSS, LabVIEW, Pascal, Arduino.
    
    \cusemph{Toolboxes}: Numba \texttt{+} SciPy \texttt{+} Numpy, PyTorch/TensorFlow.
\end{section}
\par\vspace{-2ex}
\sectionline

%-------------------------------------------------------------------------------
%       ACHIEVEMENTS && Selected Publications
%-------------------------------------------------------------------------------
\begin{section}{Awards \& Achievements}
    Awarded \cusemph{Second-class Scholarship and ``Execellent Student''} from 2016 to 2017. \\
    Awarded \cusemph{Best Track Paper of Algorithm Track} in IEEE MSN, 2020.
\end{section}
% \par\vspace{-2ex}
% \sectionline

\begin{section}{Selected Publications}
    [1] \cusemph{Y. Hong}, B. Lv, R. Wang, H. Tan, Z. Han, H. Zhou, F. C. M. Lau, ``Online Distributed Job Dispatching with Outdated and Partially-Observable Information,'' in \emph{Proc. IEEE MSN, 2020} \cusemph{Best Track Paper Award}.

    [2] \cusemph{Y. Hong}, B. Lv, R. Wang, H. Tan, Z. Han, F. C. M. Lau, ``Distributed Job Dispatching in Edge Computing Networks with Random Transmission Latency: A Low-Complexity POMDP Approach,'' in \emph{IEEE Internet of Things Journal}, doi: 10.1109/JIOT.2021.3103798.

    [3] L. Zhou, \cusemph{Y. Hong}, S. Wang, R. Han, D. Li, R. Wang, and Q. Hao, ``Learning centric wireless resource allocation for edge computing: Algorithm and experiment,'' \emph{IEEE Transactions on Vehicular Technology}, Jan. 2021.

    \vspace{-2ex}
\end{section}
\sectionline

\section{Academic Researches}
\begin{research}
  \title{Job Dispatching with Outdated-and-Partial Information in Edge Computing System}
  \supervisor{Supervisors: Prof. R. Wang, Prof. H. Tan, and Prof. F. C. M. Lau}
  \duration{Nov '18 - Feb '21}
  \description{
    -- We formulated the distributed and cooperative job dispatching in edge computing system with outdated and partial information as a POMDP problem. \\
    -- We proposed a novel low-complexity approximate MDP solution framework via alternative policy iteration, and derived an analytical performance lower bound and a tighter semi-analytical lower bound. \\
    -- We conducted extensive simulations based on the Google Cluster trace, and compare our approach with three
    heuristic benchmarks. The evaluation results show that our proposed algorithm can achieve as high as $20.67\%$ reduction in average job response time and consistently perform well under various parameter settings of information staleness.
  }
\end{research}
\newpage

\begin{research}
  \title{Online Federated Learning on Mobile Vehicles for 3D Object Detection}
  \supervisor{Supervisors: Prof. R. Wang and Prof. F. C. M. Lau }
  \duration{Nov '19 - Present}
  \description{
    -- We defined a scenario where multiple vehicles equipped with a SECOND network cooperatively collect the 3D object information to the edge server. \\
    -- We developed a semi-supervised federated learning framework, where the fused knowledge at the edge server is transferred to the vehicles with knowledge distillation for model accuracy improvement. \\
    -- We implemented the experiment based on a high-fidelity autonomous driving simulator CARLA, and generated some datasets conformed to the KITTI dataset format. The source code is available on \href{https://github.com/zijianzhang/CARLA_INVS}{Github}. \\
    -- In the extended work, we formulate a resource allocation problem in the above scenario, to achieve both communication and computation efficiency (Hybrid Utilization Maximization) in online federated training.
  }
\end{research}

\vspace{-2ex}
\sectionline
%-------------------------------------------------------------------------------


%-------------------------------------------------------------------------------
%	ACADEMIC PROJECTS SECTION
%-------------------------------------------------------------------------------
\begin{section}{Projects}      
    \begin{project}
        \title{Manipulation of Networking Stack in Linux Kernel}
        \duration{Oct '16 - Oct '18}
        \description{
          -- \cusemph{MAC Layer (L2)}: (\href{http://github.com/iamhyc/wlsops-hack}{Github Link})
          \begin{itemize}
            \item[-] This project aims at providing a faster and reliable access to the wireless NIC driver from userspace.
            % In fact, Linux mac80211 subsystem provides a set of APIs to monitor and modify the parameters related to IEEE 802.11 MAC design. The existing tools rely on either ``ioctl'' or ``genl'' which could not provide real-time and robust access to kernel driver.
            We design one kernel module to \cusemph{hijack the function entry} of the driver, and use ``mmap'' to establish a shared-memory communication between kernel and userspace.
            \item[-] The experiment results show that we could alter a set of \cusemph{IEEE 802.11e} parameters (related to the channel access priority) within 10 millisecond for 1000 accesses. 
          \end{itemize}
          -- \cusemph{IP Layer (L3)}: (\href{http://github.com/iamhyc/Netfilter-L4-Encryption}{Github Link})
          \begin{itemize}
            \item[-] This project implements an in-stack network data encryption trial based on Linux Netfilter subsystem.
            It uses asynchronous encryption method \cusemph{AES-128} provided by Linux kernel, to encrypt the payload of L3 (i.e., content of an IP packet) and decrypt it correspondingly at the receiver’s side.
            \item[-] This project would be extended with an identification mechanism which allows the tx/rx to negotiate the enabling of en(de)cryption, and a userspace tool based on \cusemph{Linux Netlink subsystem} for key management. 
          \end{itemize}
        }
    \end{project}
    \\
    \begin{project}
      \title{VLC-WiFi Integrated Communication Platform}
      %\links{\href{https://github.com/rbavishi/iCBMC}{Github Link}}
      \duration{Sep '17 - Dec '18}
      \description{
        -- In this project, we implemented a {hybrid VLC-WiFi} communication system, where the VLC link bears the high-throughput downlink, and Wi-Fi link servers as uplink. \\
        -- We implemented the system on \cusemph{NI instruments} with out-of-shelf \cusemph{Wi-Fi NIC}, and the retransmission mechanism is implemented over IP layer. The source code of the system implementation is provided on \href{https://github.com/iamhyc/CSDS-VLC-TCP}{Github}. We also have one granted patent (CN110429979B) over this platform.
      }
    \end{project}
    \\
    \begin{project}
      \title{SerDe-based Inter-Process Communication (IPC) Framework}
      \duration{Apr '21 - Present}
      \description{
        -- This IPC framework is writen in Rust language and the source code is available on \href{https://github.com/VDM-Maintainer-Group/vdm-capability-library}{Github}. (\cusemph{SerDe} stands for serializing and deserializing) \\ %, which is also the name of a crate developed in Rust
        -- This IPC framework borrows the paradigms from \emph{Android Binder} and \emph{gRPC}, which aims at providing unified FFI access for \cusemph{any typed-function signatures in any languages}. It currently supports Rust/CPP/Python services. \\
        -- This IPC framework is designed in C/S model, and the backend is in modular design. More specifically, it could support different IPC protocols (e.g., socket, binder) and different {SerDe} methods (e.g., json, ProtoBuf). \\
        % - This IPC framework currently serves as a component of \cusemph{VDM Capability Library} which aims at gathering all best system-related practices. 
      }
  \end{project}

\end{section}

% \vspace{-2ex}
% \sectionline
%-------------------------------------------------------------------------------


\end{resume}
\end{document}

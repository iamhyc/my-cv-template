%%%%%%%%%%%%%%%%%%%%%%%%%%%%%%%%%%%%%%%%%%%%%%%%%%%%%%%%%%%%%%%%%%%%%%%%%%%%%%%%
% Medium Length Graduate Curriculum Vitae
% LaTeX Template
% Version 1.2 (3/28/15)
%
% This template has been downloaded from:
% http://www.LaTeXTemplates.com
%
% Original author:
% Rensselaer Polytechnic Institute 
% (http://www.rpi.edu/dept/arc/training/latex/resumes/)
%
% Modified by:
% Daniel L Marks <xleafr@gmail.com> 3/28/2015
% 
% Further modified by:
% Rohan Bavishi <rohan.bavishi95@gmail.com> 9/20/2016
%
% Important note:
% This template requires the simple_style.cls file to be in the same directory 
% as the .tex file. The res.cls file provides the resume style used for 
% structuring the document.
%
%%%%%%%%%%%%%%%%%%%%%%%%%%%%%%%%%%%%%%%%%%%%%%%%%%%%%%%%%%%%%%%%%%%%%%%%%%%%%%%%

%-------------------------------------------------------------------------------
%	PACKAGES AND OTHER DOCUMENT CONFIGURATIONS
%-------------------------------------------------------------------------------

%%%%%%%%%%%%%%%%%%%%%%%%%%%%%%%%%%%%%%%%%%%%%%%%%%%%%%%%%%%%%%%%%%%%%%%%%%%%%%%%
% You can have multiple style options the legal options ones are:
%
%   centered:	the name and address are centered at the top of the page 
%				(default)
%
%   line:		the name is the left with a horizontal line then the address to
%				the right
%
%   overlapped:	the section titles overlap the body text (default)
%
%   margin:		the section titles are to the left of the body text
%		
%   11pt:		use 11 point fonts instead of 10 point fonts
%
%   12pt:		use 12 point fonts instead of 10 point fonts
%
%%%%%%%%%%%%%%%%%%%%%%%%%%%%%%%%%%%%%%%%%%%%%%%%%%%%%%%%%%%%%%%%%%%%%%%%%%%%%%%%

\documentclass[mm]{simple_style}  

% Default font is the helvetica postscript font
\usepackage{helvet}
\usepackage{hyperref}
\usepackage{url}
\usepackage{xcolor}
\hypersetup {
    colorlinks=true,
    linkcolor=colorlink,
    filecolor=magenta,      
    urlcolor=colorlink,
}
\usepackage[left=0.7in, right=2in, top=0.9in]{geometry}

% Increase text height
\textheight=700pt

\begin{document}

%-------------------------------------------------------------------------------
%	NAME AND ADDRESS SECTION
%-------------------------------------------------------------------------------
\name{Yuncong Hong}
\qualification{PhD Candidate, Computer Science, The University of Hong Kong}
\emailone{ychong@cs.hku.hk}
\emailtwo{hongyc@mail.sustech.edu.cn}
\website{http://iamhyc.github.io}{ \url{iamhyc.github.io} }
\github{https://github.com/iamhyc}{ \url{github.com/iamhyc} }
\phone{+86 15811823462}
\address{
    1088 Xueyuan Avenue, \\
    Shenzhen 518055, P.R. China
}
%-------------------------------------------------------------------------------

\begin{resume}

%-------------------------------------------------------------------------------
%	EDUCATION SECTION
%-------------------------------------------------------------------------------
\begin{section}{Education}
    \cusemph{Southern University of Science and Technology (SUSTech)} \\ %, China
    {\sl Bachelor of Engineering} in \cusemph{Communication Engineering}, \timeline{Sep '14 - Jul '18}\\
    \cusemph{GPA: 3.76/4.00} (Overall)~\cusemph{Ranking: 3/30}
    
    \cusemph{The University of Hong Kong, Hong Kong} \\ %, Chine
    {\sl Doctor of Philosophy} in \cusemph{Computer Science}, \timeline{Sep '18 - Jul '22} (expected)\\
\end{section}
\sectionline

%-------------------------------------------------------------------------------
%      RESEARCH INTERESTS && SKILLS 
%-------------------------------------------------------------------------------
\begin{section}{Research Interests}
  Edge Computing, Application-Centric Network Optimization, and Applications of Markov Decision Process (MDP).
\end{section}

\begin{section}{Research Skills}
    \cusemph{Core Courses}: Convex Optimization, Polynomial Optimization, Stochastic Process, Wireless Communications, and Information Theory.
    
    \cusemph{Programming Languages}: Python, Rust, C/C++, Nodejs, TypeScript, and LabVIEW.
    % \LaTeX, Java, HTML/(S)CSS, LabVIEW, Pascal, Arduino.
    
    \cusemph{Toolboxes}: Numba + SciPy + Numpy, PyTorch/TensorFlow.
\end{section}
\par\vspace{-2ex}
\sectionline

%-------------------------------------------------------------------------------
%       ACHIEVEMENTS && Selected Publications
%-------------------------------------------------------------------------------
\begin{section}{Awards \& Achievements}
    Awarded \cusemph{Second-class Scholarship and ``Execellent Student''} from 2016 to 2017. \\
    Awarded \cusemph{Best Track Paper of Algorithm Track} in IEEE MSN, 2020.
\end{section}
% \par\vspace{-2ex}
% \sectionline

\begin{section}{Selected Publications}
    [1] \cusemph{Y. Hong}, B. Lv, R. Wang, H. Tan, Z. Han, H. Zhou, F. C. M. Lau, ``Online Distributed Job Dispatching with Outdated and Partially-Observable Information,’’ in \emph{Proc. IEEE MSN, 2020} \cusemph{Best Track Paper Award}.

    [2] \cusemph{Y. Hong}, B. Lv, R. Wang, H. Tan, Z. Han, F. C. M. Lau, ``Distributed Job Dispatching in Edge Computing Networks with Random Transmission Latency: A Low-Complexity POMDP Approach,’’ in \emph{IEEE Internet of Things Journal}, doi: 10.1109/JIOT.2021.3103798.

    [3] L. Zhou, \cusemph{Y. Hong}, S. Wang, R. Han, D. Li, R. Wang, and Q. Hao, “Learning centric wireless resource allocation for edge computing: Algorithm and experiment,” \emph{IEEE Transactions on Vehicular Technology}, Jan. 2021.

    \vspace{-2ex}
\end{section}
\sectionline

\section{Academic Researches}
\begin{research}
  \title{Job Dispatching with Outdated-and-Outdated Information in Edge Computing System}
  \supervisor{Supervisors : Prof. R. Wang, Prof. H. Tan, and Prof. F. C. M. Lau}
  \duration{Nov '18 - Feb '21}
  \description{
    - Stale and Outdated Information, POMDP problem formulation \\
    - alternative update solution framework \\
    - numerical simulations with Google cluster data trace
  }
\end{research}
\newpage

\begin{research}
  \title{Online Federated Learning on Mobile Vehicles for 3D Object Detection}
  \supervisor{Supervisors : Prof. R. Wang and Prof, F. C. M. Lau }
  \duration{Nov '19 - Present}
  \description{
    - High-fidelity simulation platform CARLA \\
    - Semi-supervised federated learning with knowledge distillation \\
    - multiple metric compared, datasets generated, and open-source code with followers \\
    - Ongoing: Resource allocation for energy-efficient online learning.
  }
\end{research}

\vspace{-2ex}
\sectionline
%-------------------------------------------------------------------------------


%-------------------------------------------------------------------------------
%	ACADEMIC PROJECTS SECTION
%-------------------------------------------------------------------------------
%FIXME: modify and fill-in
\begin{section}{Projects}
    \begin{project}
        \title{SerDe IPC Framework} % VDM
        \duration{Apr '14 - Present}
        \description{
          - Used in VDM \\
          - Written in Rust lang (with safety) \\
          - Unified FFI accesses for dynamic libraries for C/CPP and Rust and Libraries for Python \\
          - Decoupled Design for 
          % - Re-invented a two-pass \textit{deterministic} algorithm to find the median of large data-sets (approx. 1 TB)\\
          % - The algorithm developed was similar to the one described in the \href{http://polylogblog.files.wordpress.com/2009/08/80munro-median.pdf}{paper} by Munro-Paterson (1980)\\
          % - Carried out extensive tests to evaluate the performance of the algorithm\\
          % - \href{https://drive.google.com/file/d/0B0--s-r8CTxgZTU3RzB3YnhMMVU/view}{Report}
        }
    \end{project}
        
    \begin{project}
        \title{Manipulation of (Wireless) Network Stacks in Linux Kernel} %Netfilter + Netlink + mac80211
        \duration{Oct '16 - Oct '18}
        \description{
            % - A linux application for backing-up and syncing files between two or more peers\\
            % - Users have a shared folder across different machines, with local copies. Changes made in any one copy are synced across all devices\\
            % - Linux \emph{inotify} API used to track changes in the shared folder and \emph{rsync} used to sync the modifications to ensure efficient transfer\\
            % - Multithreading with mutexes used to parallelize syncing and file-monitoring operations\\
            % - \href{https://github.com/rbavishi/P2P-Dropbox}{Github Link}
        }
    \end{project}

    \begin{project}
      \title{VLC-WiFi integrated Network}
      %\links{\href{https://github.com/rbavishi/iCBMC}{Github Link}}
      \duration{Sep '17 - Dec '18}
      \description{
        In this project, we implemented a hybrid VLC-WiFi communication system, where the VLC link bears the high-throughput downlink, and Wi-Fi link servers as uplink. We implemented the system on NI instruments with out-of-shelf Wi-Fi NIC, and the retransmission mechanism is implemented on IP layer. The source code of the whole workable system implementation is provided here.
        % The granted patent is 数据传输方法、装置、设备及存储介质 (CN110429979B).
        % - Ported the described \emph{Component-Based-Synthesis} algorithm in \href{https://www.comp.nus.edu.sg/~abhik/pdf/ICSE15-directfix.pdf}{DirectFix} to CBMC\\
        % - Reproduced the experimental results provided in the paper, and devised further optimizations\\
        % - \href{https://github.com/rbavishi/iCBMC}{Github Link}
      }
    \end{project}

\end{section}

% \vspace{-2ex}
% \sectionline
%-------------------------------------------------------------------------------


\end{resume}
\end{document}

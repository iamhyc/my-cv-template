%%%%%%%%%%%%%%%%%%%%%%%%%%%%%%%%%%%%%%%%%%%%%%%%%%%%%%%%%%%%%%%%%%%%%%%%%%%%%%%%
% Medium Length Graduate Curriculum Vitae
% LaTeX Template
% Version 1.2 (3/28/15)
%
% This template has been downloaded from:
% http://www.LaTeXTemplates.com
%
% Original author:
% Rensselaer Polytechnic Institute 
% (http://www.rpi.edu/dept/arc/training/latex/resumes/)
%
% Modified by:
% Daniel L Marks <xleafr@gmail.com> 3/28/2015
% 
% Further modified by:
% Rohan Bavishi <rohan.bavishi95@gmail.com> 9/20/2016
%
% Important note:
% This template requires the simple_style.cls file to be in the same directory 
% as the .tex file. The res.cls file provides the resume style used for 
% structuring the document.
%
%%%%%%%%%%%%%%%%%%%%%%%%%%%%%%%%%%%%%%%%%%%%%%%%%%%%%%%%%%%%%%%%%%%%%%%%%%%%%%%%

%-------------------------------------------------------------------------------
%	PACKAGES AND OTHER DOCUMENT CONFIGURATIONS
%-------------------------------------------------------------------------------

%%%%%%%%%%%%%%%%%%%%%%%%%%%%%%%%%%%%%%%%%%%%%%%%%%%%%%%%%%%%%%%%%%%%%%%%%%%%%%%%
% You can have multiple style options the legal options ones are:
%
%   centered:	the name and address are centered at the top of the page 
%				(default)
%
%   line:		the name is the left with a horizontal line then the address to
%				the right
%
%   overlapped:	the section titles overlap the body text (default)
%
%   margin:		the section titles are to the left of the body text
%		
%   11pt:		use 11 point fonts instead of 10 point fonts
%
%   12pt:		use 12 point fonts instead of 10 point fonts
%
%%%%%%%%%%%%%%%%%%%%%%%%%%%%%%%%%%%%%%%%%%%%%%%%%%%%%%%%%%%%%%%%%%%%%%%%%%%%%%%%

\documentclass[mm]{simple_style_cn}  

% Default font is the helvetica postscript font
\usepackage{helvet}
\usepackage{hyperref}
\usepackage{url}
\usepackage{xcolor}
\hypersetup {
    colorlinks=true,
    linkcolor=colorlink,
    filecolor=magenta,      
    urlcolor=colorlink,
}
\usepackage[left=0.7in, right=2in, top=0.9in]{geometry}

% Increase text height
\textheight=700pt

\begin{document}

%-------------------------------------------------------------------------------
%	NAME AND ADDRESS SECTION
%-------------------------------------------------------------------------------
\name{洪云聪}
\qualification{PhD Candidate, Computer Science, The University of Hong Kong}
\emailone{ychong@cs.hku.hk}
% \emailtwo{hongyc@mail.sustech.edu.cn}
\website{https://iamhyc.github.io}{ \url{iamhyc.github.io} }
\github{https://github.com/iamhyc}{ \url{github.com/iamhyc} }
\phone{\texttt{+}86 15811823462}
\address{
    % 1088 Xueyuan Avenue, \\
    % Shenzhen 518055, P.R. China
}
%-------------------------------------------------------------------------------

\begin{resume}

%-------------------------------------------------------------------------------
%	EDUCATION SECTION
%-------------------------------------------------------------------------------
\begin{section}{教育背景}
    \cusemph{南方科技大学 (SUSTech)} \\ %, China
    \cusemph{通信工程} - {\sl 学士} \timeline{2014 - 2018}{~~~~~~~~~~}\\
    \cusemph{专业GPA: 3.77}/4.00 ~ \cusemph{排名: 3/30}.
    
    \cusemph{香港大学 (HKU)} \\ %, Hong Kong, China
    \cusemph{计算机科学} - {\sl 博士} ~ \timeline{2018 - 2022} (预计)\\
\end{section}
\vspace{-2ex}
\sectionline

%-------------------------------------------------------------------------------
%      RESEARCH INTERESTS && SKILLS 
%-------------------------------------------------------------------------------
\begin{section}{研究方向}
  边缘计算,面向应用(Application-centric)的网络优化,马尔科夫决策过程(MDP)应用。
  % Edge Computing, Application-Centric Network Optimization, and Applications of Markov Decision Process (MDP).
\end{section}

\begin{section}{科研技能}
    \cusemph{核心课程}: 凸优化,多项式优化,随机过程;无线通信,信息论。
    \\
    \cusemph{编程语言}: Python, Rust, C/C\texttt{++}, Nodejs, TypeScript, and LabVIEW。
    \\
    % \LaTeX, Java, HTML/(S)CSS, LabVIEW, Pascal, Arduino.
    \cusemph{仿真工具}: Numba \texttt{+} SciPy \texttt{+} Numpy, PyTorch/TensorFlow。
\end{section}
\par\vspace{-2ex}
\sectionline

%-------------------------------------------------------------------------------
%       ACHIEVEMENTS && Selected Publications
%-------------------------------------------------------------------------------
\begin{section}{个人成就}
    Awarded \cusemph{Second-class Scholarship and ``Execellent Student''} from 2016 to 2017. \\
    Awarded \cusemph{Best Track Paper of Algorithm Track} in IEEE MSN, 2020.
\end{section}
\vspace{-2ex}
% \sectionline

\begin{section}{代表论文}
    [1] \cusemph{Y. Hong}, B. Lv, R. Wang, H. Tan, Z. Han, H. Zhou, F. C. M. Lau, ``{Online Distributed Job Dispatching with Outdated and Partially-Observable Information},'' in \emph{Proc. IEEE MSN, 2020} \cusemph{Best Track Paper Award}. \newline
    [2] \cusemph{Y. Hong}, B. Lv, R. Wang, H. Tan, Z. Han, F. C. M. Lau, ``{Distributed Job Dispatching in Edge Computing Networks with Random Transmission Latency: A Low-Complexity POMDP Approach},'' in \emph{IEEE Internet of Things Journal}. \newline
    [3] L. Zhou, \cusemph{Y. Hong}, S. Wang, R. Han, D. Li, R. Wang, and Q. Hao, ``{Learning centric wireless resource allocation for edge computing: Algorithm and experiment},'' \emph{IEEE Transactions on Vehicular Technology}, Jan. 2021.
    \vspace{-2ex}
\end{section}

\sectionline

\section{学术研究}
\begin{research}
  \title{基于部分与过时信息(Outdated-and-Partial Information)的边缘计算任务调度}
  \supervisor{指导导师:王锐,谈海生,Prof. Francis C. M. Lau。}
  \duration{Nov '18 - Feb '21}
  \description{
    -- 在这个课题中,我们将边缘计算系统中,具有过时和部分信息的分布式任务调度问题,表述为一个POMDP问题。 \\
    -- 为了解决MDP求解过程中的维度爆炸(curse of dimensionality)问题,我们提出了一个低复杂度的解决方案,替代策略迭代(alternative policy iteration),并得出了一个解析性能下限和一个更严格的半解析下限。\\
    -- 基于Google Cluster的公开数据集,我们进行了大量的仿真并将我们的方法和三个启发式算法进行比较。结果显示,我们提出的算法可以实现减少高达$20.67\%$的平均响应时间,并且在信息滞后性的各种参数设置下始终表现良好。
  }
\end{research}
\newpage

\begin{research}
  \title{在移动车辆上部署用于3D物体检测的在线联邦学习}
  \supervisor{指导导师:王锐,Prof. Francis C. M. Lau。}
  \duration{Nov '19 - 现在}
  \description{
    -- 在这个课题中,我们首先给出了一个场景,多个配备SECOND网络的车辆将会合作收集道路上的3D物体信息到边缘服务器。\\
    -- 在这个应用场景下,我们开发了一个半监督(semi-supervised)的联合学习框架,边缘服务器上的融合去噪的物体信息返回给车辆,车辆可以通过这种知识提炼(Knowledge Distillation)的方法进一步提高模型精度。\\
    -- 我们在高保真自动驾驶模拟器CARLA的基础上进行了实验,并生成了符合KITTI数据集格式的数据集。 源代码可以在Github上找到 \href{https://github.com/zijianzhang/CARLA_INVS}{Github}. \\
    -- 在后续的扩展工作中,我们提出了上述场景下的资源分配问题,以实现在线联邦学习中的通信和计算效率最大化。
  }
\end{research}
\par\vspace{-7ex}
\sectionline
%-------------------------------------------------------------------------------

%-------------------------------------------------------------------------------
%	ACADEMIC PROJECTS SECTION
%-------------------------------------------------------------------------------
\begin{section}{项目经历}      
    \begin{project}
        \title{Linux内核中网络堆栈的实践}
        \duration{Oct '16 - Oct '18}
        \description{
          -- \cusemph{MAC Layer (L2)}: (\href{http://github.com/iamhyc/wlsops-hack}{Github Link})
            在这个项目中,我们提供了对于无线网卡驱动控制更快、更可靠的访问。
            % In fact, Linux mac80211 subsystem provides a set of APIs to monitor and modify the parameters related to IEEE 802.11 MAC design. The existing tools rely on either ``ioctl'' or ``genl'' which could not provide real-time and robust access to kernel driver.
            我们设计了一个内核模块来劫持驱动程序的函数入口,并使用``mmap''来建立内核和用户空间之间的共享内存通信。
            实验结果表明,对于一组\cusemph{IEEE 802.11e}参数(与信道访问优先级有关)的1000次操作,我们可以在10ms内完成。
          \newline
          -- \cusemph{IP Layer (L3)}: (\href{http://github.com/iamhyc/Netfilter-L4-Encryption}{Github Link})
            该项目实现了基于Linux Netfilter子系统的网络数据加解密试验(类似于IPsec)。它使用Linux内核提供的异步加密方法AES-128,对L3的有效载荷(即IP数据包的内容)进行加密,并在接收方进行相应解密。这个项目的扩展工作中会实现一个识别机制,允许Tx/Rx协商启用加解密,以及添加基于Linux Netlink子系统的用户空间工具来管理密钥。
        }
    \end{project}
    \\
    \begin{project}
      \title{VLC-WiFi 融合通信平台}
      %\links{\href{https://github.com/rbavishi/iCBMC}{Github Link}}
      \duration{Sep '17 - Dec '18}
      \description{
        在这个项目中,我们实现了一个混合VLC-WiFi通信系统,其中VLC链接承担高吞吐量的下行链路,Wi-Fi链接服务器作为上行链路。
        我们基于\cusemph{NI仪器}和USB Wi-Fi网卡实现了该系统,其中重传机制通过IP层实现。系统实现的源代码在\href{https://github.com/iamhyc/CSDS-VLC-TCP}{Github}上提供。该平台上还拥有一项授权专利(CN110429979B)产出。
      }
    \end{project}
    \\
    \begin{project}
      \title{基于SerDe机制的IPC(Inter-Process Communication)框架}
      \duration{Apr '21 - Present}
      \description{
        -- 该IPC框架使用Rust语言编写,源代码可以在\href{https://github.com/VDM-Maintainer-Group/vdm-capability-library}{Github}上找到。其中``SerDe''是``serializing-and-deserializing''的意思。\\
        %, which is also the name of a crate developed in Rust
        -- 该IPC借鉴了来自\emph{Android Binder}和\emph{gRPC}的优秀实践,旨在提供对于\cusemph{任何语言中任意函数签名}的统一FFI访问(需要有稳定的ABI),目前仅支持 Rust/CPP/Python 的服务。\\
        -- 该IPC框架采用C/S模型设计,并且后端是模块化的。理论上它可以支持不同的\cusemph{IPC协议}(例如socket,binder)和不同的\cusemph{SerDe机制}(例如json,ProtoBuf)。
        % - This IPC framework currently serves as a component of \cusemph{VDM Capability Library} which aims at gathering all best system-related practices. 
      }
  \end{project}

\end{section}

% \vspace{-2ex}
% \sectionline
%-------------------------------------------------------------------------------


\end{resume}
\end{document}
